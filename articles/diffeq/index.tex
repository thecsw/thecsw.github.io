% Created 2021-04-20 Tue 13:48
% Intended LaTeX compiler: xelatex
\documentclass[12pt]{article}
\usepackage{graphicx}
\usepackage{grffile}
\usepackage{longtable}
\usepackage{wrapfig}
\usepackage{rotating}
\usepackage[normalem]{ulem}
\usepackage{amsmath}
\usepackage{textcomp}
\usepackage{amssymb}
\usepackage{capt-of}
\usepackage{hyperref}
\usepackage{minted}
\usepackage{amsmath}
\usepackage{amssymb}
\usepackage{setspace}
\usepackage{subcaption}
\usepackage{mathtools}
\usepackage{xfrac}
\usepackage[margin=1in]{geometry}
\usepackage{marginnote}
\usepackage[utf8]{inputenc}
\usepackage{color}
\usepackage{epsf}
\usepackage{tikz}
\usepackage{graphicx}
\usepackage{pslatex}
\usepackage{hyperref}

\usepackage{concmath}
\usepackage[OT1]{fontenc}

\usepackage{textgreek}
\renewcommand*{\textgreekfontmap}{%
{phv/*/*}{LGR/neohellenic/*/*}%
{*/b/n}{LGR/artemisia/b/n}%
{*/bx/n}{LGR/artemisia/bx/n}%
{*/*/n}{LGR/artemisia/m/n}%
{*/b/it}{LGR/artemisia/b/it}%
{*/bx/it}{LGR/artemisia/bx/it}%
{*/*/it}{LGR/artemisia/m/it}%
{*/b/sl}{LGR/artemisia/b/sl}%
{*/bx/sl}{LGR/artemisia/bx/sl}%
{*/*/sl}{LGR/artemisia/m/sl}%
{*/*/sc}{LGR/artemisia/m/sc}%
{*/*/sco}{LGR/artemisia/m/sco}%
}
\makeatletter
\newcommand*{\rom}[1]{\expandafter\@slowromancap\romannumeral #1@}
\makeatother
\DeclarePairedDelimiterX{\infdivx}[2]{(}{)}{%
#1\;\delimsize\|\;#2%
}
\newcommand{\infdiv}{D\infdivx}
\DeclarePairedDelimiter{\norm}{\left\lVert}{\right\rVert}
\DeclarePairedDelimiter{\ceil}{\left\lceil}{\right\rceil}
\DeclarePairedDelimiter{\floor}{\left\lfloor}{\right\rfloor}
\def\Z{\mathbb Z}
\def\R{\mathbb R}
\def\C{\mathbb C}
\def\N{\mathbb N}
\def\Q{\mathbb Q}
\def\noi{\noindent}
\onehalfspace
\usemintedstyle{bw}
\author{Sandy Urazayev\thanks{thecsw@Mania.local}}
\date{\today}
\title{}
\hypersetup{
 pdfauthor={Sandy Urazayev},
 pdftitle={},
 pdfkeywords={},
 pdfsubject={},
 pdfcreator={Emacs 28.0.50 (Org mode 9.4.5)}, 
 pdflang={English}}
\begin{document}

\section*{Differential Equations Cookbook 🔥}
\label{sec:org7df5a1d}

May 12th, 2020

\href{./index.pdf}{(PDF Version)}

\subsection*{Abstract}
\label{sec:org60db1d2}

When I got first introduced to differential equations, I had a love-hate
relationship with it. Mainly due to some back-of-the-book problems we were given
and never-ending projects we were assigned to. After some time, differential
equations is a way to truly understand physics and the foundations of gravity,
fields, and everything. This articles is merely an intro on manually solving
common forms of differential equations. Hope you enjoy

\subsubsection*{Quick notes}
\label{sec:org5e5e9a1}

\begin{itemize}
\item \(f_x \iff \partial_x f\)
\item \(A,B,C\) are usually constants
\item \(c_k\) is usually solution's constant that is defined with initial conditions
\item Most of the functions are \(\mathbb{R}^k \to \mathbb{R}\), \(k \in \mathbb{N}^+\)
\item If you found a typo or want to comment, feel free to email me. Email on
top of the page.
\end{itemize}


\subsubsection*{First-order, linear}
\label{sec:org12e90ac}

Those equations have the form: \(y' + p(t) y = q(t)\)

Find \(\mu(t) = e^{\int p(t) dt}\)

Then \(\frac{d}{dt}(\mu(t)y) = q(t) \mu(t)\)
\(\implies y = \frac{\int q(t) \mu(t) dt}{\mu(t)}\)

\subsubsection*{First-order, separable}
\label{sec:org1bac79d}

Those equations have the form: \(\frac{dy}{dx} = f(x)g(y)\)

Find the solution by solving \(\int \frac{dy}{g(y)} = \int f(x) dx\)

Solve for exact (explicit) values of \(y\)

\subsubsection*{Exact equations}
\label{sec:org47a1f4f}

They have the form \(M(x,y) + N(x,y) \frac{dy}{dx} = 0\)

\begin{itemize}
\item \((\xi)\) If \(M_y = N_x\)

\(\implies\) Find such \(F(x,y)=C\), where \(F_x = M\), \(F_y = N\)

\item otherwise, make it exact, such that

\(\frac{M_y-N_x}{N}\) only depends on \(x\) \textbf{or} \(\frac{N_x-M_y}{M}\) only
depends on \(y\) 

Find \(\mu(x) = e^{\int \frac{M_y-N_x}{N} dx}\) \textbf{or}
\(\mu(y) = e^{\int \frac{N_x-M_y}{M} dy}\), multiply both by \(M\) \textbf{and} \(N\),
so the condition \(M_y = N_y\) is satisfied. Go to \((\xi)\) and proceed with
finding \(F(x,y)\)
\end{itemize}

\subsubsection*{Second-order, linear, constant-coefficient equations}
\label{sec:org9f4225c}

They have the form \(y'' + p y' + q y = f(t)\)

\begin{itemize}
\item First, solve for the homogeneous case, where \(y'' + p y' + q y = 0\)

Make a characteristic polynomial, let \(y = e^{rt}\): 

\(r^2+pr+q=0\)

Find roots, solution (general) will be \(y = c_1 e^{r_1 t} + c_2 e^{r_2 t}\)

\textbf{**} if repeated root \(\implies y = c_1 e^{rt} + c_2 t e^{rt}\)

\textbf{**} if \(r = \alpha \pm i \beta\) \(\implies y=c_1 \cos(\beta t)e^{\alpha t} + c_2 \sin(\beta t) e^{\alpha t}\)

\item Solving for particular solution \(y_p(t)\)

\textbf{**} Undetermined coefficients (superpositioned) for \(f(t)\)

Whatever is in \(f(t)\), start adding up the corresponding coefficients into
\(y_p(t)\)

\textbf{\textbf{*}} \(e^{nt} \to Ae^{nt}\)

\textbf{\textbf{*}} \(t^m \to A_m t^m + \ldots + A_1 t + A_0\)

\textbf{\textbf{*}} \(\cos(\beta t)\) or \(\sin(\beta t) \to Acos(\beta t) + B\sin(\beta t)\)

NOTE: should not be linearly dependent with the general solution. If it
is, multiply by \(t\) until it is linearly independent.

\textbf{**} Variation of parameters

Seek \(y_p(t) = v_1(t)y_1(t)+v_2(t)y_2(t)\), where

\(\begin{cases}v_1'y_1+v_2'y_2=0\\v_1'y_1'+v_2'y_2'=f(t)\end{cases}\)
\end{itemize}


So the final solution is \(y(t)=c_1 y_1(t) + c_2 y_2(t) + y_p(t)\)

\subsubsection*{Second-order, linear, variable-coefficient equations}
\label{sec:orgd06c40a}

Equations have the form 

\begin{itemize}
\item \((1)\): \(a(t)y'' + b(t)y'+c(t)y = f(t)\)
\item \((2)\): \(y'' + p(t)y'+q(t)y = g(t)\)
\end{itemize}

In general case, guess the first homogeneous solution (try \(y_1=e^t\)) and
use reduction of order to find the second homogeneous solution, so that

\(y_2(t) = v(t)y_1(t)\)

\(\implies y_2'' + p(t)y_2' + q(t)y_2 = 0\)

\(\implies (v(t)y_1(t))''+p(t)(v(t)y_1(t))'+q(t)(v(t)y_1(t))=0\)

NOTE: Also applicable with form \((1)\)

You will probably have another differential equation emerge from above. It
should have lower order than our current equation, so just refer to one of
the techniques above to find \(v(t)\) and then you can find
\(y_2(t)=v(t)y_1(t)\)

Use \textbf{variation of parameters} to find a particular solution. It's that
system with \(v\)

NOTE: What you if you have a \textbf{Cauchy-Euler equation}?

They have the form \(at^2y''+bty'+cy=0\)

then \(y=t^r \implies ar^2+(b-a)r+c=0\)

\begin{itemize}
\item if \(r\) is repeated, \(y_1=t^r\), \(y_2=ln|t|t^r\)
\item if \(r=\alpha\pm i\beta\), \(y_1=t^{\alpha}\cos(\beta ln|t|)\) and
\(y_2=t^{\alpha}\sin(\beta ln|t|)\)
\end{itemize}

Generally, solution has the form \(y=c_1t^{r_1}+c_2t^{r_2}\)

\subsubsection*{Higher-order, linear equations}
\label{sec:org9c6b86b}

\(a_n(t)y^{(n)}+a_{n-1}(t)y^{(n-1)}+\ldots+a_1(t)y'+a_0(t)y=g(t)\)

All second-order methods above extend to \(n^{th}\) order.

\subsubsection*{Laplace transform}
\label{sec:org8d7894e}

Laplace is a holy grail of solving differential equations with initial
values defined. Laplace is the same kind of Bible to engineers like Taylor
Series is. 

\(\mathcal{L}\{f\}(s) = \int_0^{\infty} e^{-st} f(t) dt\)

assuming \(f\) is piecewise continuous and of exponential order.

Table of common transformations:

\begin{center}
\begin{tabular}{ll}
\(f(t)\) & \(\mathcal{L}\{f\}(s)\)\\
\hline
\(1\) & \(\frac{1}{s}\)\\
\(e^{at}\) & \(\frac{1}{s-a}\)\\
\(\sin(bt)\) & \(\frac{b}{s^2+b^2}\)\\
\(\cos(bt)\) & \(\frac{s}{s^2+b^2}\)\\
\(u(t-a)\) & \(\frac{e^{-as}}{s}\)\\
\(\delta(t-a)\) & \(e^{-as}\)\\
\end{tabular}
\end{center}

Where \(u(t)\) is the \href{https://en.wikipedia.org/wiki/Heaviside\_step\_function}{Heaviside step function} and \(\delta(t)\) is the \href{https://en.wikipedia.org/wiki/Dirac\_delta\_function}{Dirac
delta function}.

Some Laplace transform properties:

\begin{itemize}
\item \(\mathcal{L}\{e^{at}f(t)\}(s) = \mathcal{L}\{f(t)\}(s-a)\)
\item \(\mathcal{L}\{t^nf(t)\}(s) = s^n\mathcal{L}\{f\}(s)-s^{n-1}f(0)-\ldots-sf^{(n-2)}(0)-f^{(n-1)}(0)\)
\item \(\mathcal{L}\{t^nf(t)\}(s) = (-1)^n \frac{d^n}{ds^n} \mathcal{L}\{f(t)\}(s)\)
\end{itemize}

If \(f\) is a T-periodic function, 

\(\mathcal{L}\{f(t)\}(s) = \frac{\int_0^T e^{-sT} f(t) dt}{1-e^{-sT}}\)

where \(\int_0^T e^{-sT} f(t) dt = \mathcal{L}\{f_T(t)\}(s)\), the sum of
integrals of different parts of the piecewise function.

Convolutions:

\begin{itemize}
\item \((f*g)(t) = \int_0^t f(t-v)g(v)dv\)
\item \(\mathcal{L}\{(f*g)(t)\} = \mathcal{L}\{f(t)\}(s)\cdot \mathcal{L}\{g(t)\}(s)\)
\item \((f*g)(t) = \mathcal{L}^{-1}\{F\cdot G\}(t)\), where
\(F=\mathcal{L}\{f\}(s)\) and \(G=\mathcal{L}\{g\}(s)\)
\end{itemize}

Heaviside/unit step function:

\begin{itemize}
\item \(\mathcal{L}\{u(t-a)f(t)\}(s) = e^{-as}\mathcal{L}\{f(t+a)\}(s)\)
\item \(\mathcal{L}^{-1}\{e^{-as}F(s)\}(t)=u(t-a)\mathcal{L}^{-1}\{F(s)\}(t-a)\)
\end{itemize}

If IVP is not at 0, define some new function like \(w(t)=y(t+\alpha)\), and
solve for \(w\). Finally, you can offset to find \(y\)

Step (block) function:

\begin{itemize}
\item \(\Pi_{a,b}(t) = u(t-a)-u(t-b)\)
\item \(\mathcal{L}\{\Pi_{a,b}(t)\}(s)=\frac{e^{-sa}-e^{-sb}}{s}\)
\end{itemize}

\subsubsection*{Constant-coefficient, homogeneous systems of ODE}
\label{sec:org19a6970}

\(\vec{x}' = A \vec{x}\), where \(A\in\mathbb{R}^{n\times n}\), \(x\in\mathbb{R}^n\)

If \(A\) has n linearly independent eigenvectors \(\vec{u_i}\) associated to n
eigenvalues \(\lambda_i\), then a general solution of the system is given by
\(\vec{x}(t) = c_1 e^{\lambda_1 t}\vec{u_1}+c_2e^{\lambda_2t}\vec{u_2} + \ldots + c_ne^{\lambda_nt}\vec{u_n}\)

\begin{itemize}
\item If \(\lambda=\alpha \pm i \beta\), so \(\vec{u}=\vec{a}+i\vec{b}\), we have
\end{itemize}
\(\vec{x}=c_1e^{\alpha t}(\cos(\beta t)\vec{a}-\sin(\beta t)\vec{b}) + c_2e^{\alpha t}(\cos(\beta t)\vec{b}+\sin(\beta t)\vec{a})\)

\begin{itemize}
\item Matrix exponential
\end{itemize}

\(e^{At} = \sum_{k=0}^{\infty} \frac{A^k t^k}{k!}\), where \(A^0=I\), an
identity matrix.

\begin{itemize}
\item Find solutions for any eigenvalues
\end{itemize}

\begin{itemize}
\item Compute the characteristic polynomial \(p(\lambda)\) of \(A\)
\label{sec:orga9b6e4b}

\(p(\lambda)=det(A-\lambda I)\)

\item Factor \(p(\lambda)\) into linear factors to yield
\label{sec:orgc2feade}

\(p(\lambda) = c(\lambda-\lambda_1)^{m_1} \cdot \ldots \cdot (\lambda-\lambda_k)^{m_k}\), where \(c=\pm 1\)

\item For each \(\lambda_j\), find \(m_j\) linearly independent generalized eigenvectors \(\{\vec{u_j}^{m_1},\cdots,\vec{u_j}^{m_j}\}\) satisfying
\label{sec:org29c362e}

\((A-\lambda_i I)^{m_j} \vec{u} = \vec{0}\)

\item For each \(\vec{u_j}^i\) computed in the previous step, compute \(e^{At}\vec{u_j}^i\) by
\label{sec:orgef5557a}

\(e^{At}\vec{u_j}^i\)

\(=e^{\lambda_jt}e^{(A-\lambda_jI)t}\vec{u_j}^i\)

\(=e^{\lambda_jt}(\vec{u_j}^i+t(A-\lambda_jI)\vec{u_j}^i+\cdots+\frac{t^{m_j-1}}{(m_j-1)!}(A-\lambda_jI)^{m_j-1}\vec{u_j}^i)\)
\end{itemize}

\subsubsection*{Linear systems of ODE}
\label{sec:org6f2c55d}

\(\vec{x}' = A(t)\vec{x} + \vec{f}(t)\), where \(A\in\mathbb{R}^{n\times n}\),
\(x\in\mathbb{R}^n\), \(f\in\mathbb{R}^n\)

If \(X(t)\) is a matrix whose columns are made up of n linearly independent
homogeneous solutions (\(X(t)\) is the fundamental matrix), then a general
solution may be written as \(\vec{x}(t_0)=\vec{x_0}\)

\(\vec{x}(t) = X(t)X^{-1}(t_0)\vec{x_0}+X(t)\int_{t_0}^{t}X^{-1}(s)f(s)ds\)

If \(A(t)\) is constant-coefficient, then we recover Duhamel's formula:

\(\vec{x}(t) = e^{A(t-t_0)}x_0 + \int_{t_0}^{t}e^{A(t-s)}f(s)ds\)

\subsubsection*{Applications}
\label{sec:orgdcd2933}

There are many applications of differential equations in classical
mechanics, fields, etc. Below you will find just a snippet of some very
common Physics 1/2 scenarios


\begin{itemize}
\item Falling object
\label{sec:orga5cb8b6}

\(m\frac{dv}{dt}=mg-bv\), where \(b\) is the air resistance

\item Fluid mix, define \(R_{in}\) and \(R_{out}\)
\label{sec:org6ab8918}

\(\frac{dx}{dt}=R_{in}-R_{out}\)

\item Mass-Spring System
\label{sec:org092a33c}

\begin{itemize}
\item Vertical spring (direction of gravity)
\label{sec:org77cc09c}

\(my''=-by'-k(L+y)+mg+F_{ext}(t)\), assume \(KL=mg\), where \(b\) is dumping, and \(k\) is stiffness

\item Horizontal spring
\label{sec:orgb0a8d0e}

\(my''=-by'-ky+F_{ext}(t)\), where \(b\) is dumping, and \(k\) is stiffness
\end{itemize}
\end{itemize}

\subsubsection*{Conclusion}
\label{sec:org1df014d}

This is as much as I can recover from my initial experience with differential
equations. This article is not as much to teach you how to solve them but
provide a quick lookup cheatsheet if needed or glance at different forms that we
can actually solve! There are infinitely many differential equations that we
cannot find an exact solution for!
\end{document}